\documentclass[a4paper]{report}
\usepackage[utf8]{inputenc}
\usepackage[portuguese]{babel}
\usepackage{hyperref}
\usepackage{a4wide}
\hypersetup{pdftitle={GestVendas},
pdfauthor={João Teixeira, Emanuel Rodrigues, José Ferreira},
colorlinks=true,
urlcolor=blue,
linkcolor=black}
\usepackage{subcaption}
\usepackage[cache=false]{minted}
\usepackage{listings}
\usepackage{booktabs}
\usepackage{multirow}
\usepackage{appendix}
\usepackage{tikz}
\usepackage{authblk}
\usetikzlibrary{positioning,automata,decorations.markings}

\begin{document}

\title{GestVendas\\ 
\large Grupo Nº 56}
\author{João Teixeira (A85504) \and Emanuel Rodrigues (A84776) \and José Ferreira (A83683)}
\date{\today}

\begin{center}
    \begin{minipage}{0.75\linewidth}
        \centering
        \includegraphics[width=0.4\textwidth]{eng.jpeg}\par\vspace{1cm}
        \vspace{1.5cm}
        \href{https://www.uminho.pt/PT}
        {\color{black}{\scshape\LARGE Universidade do Minho}} \par
        \vspace{1cm}
        \href{https://www.di.uminho.pt/}
        {\color{black}{\scshape\Large Departamento de Informática}} \par
        \vspace{1.5cm}
        \maketitle
    \end{minipage}
\end{center}

\tableofcontents

\pagebreak

\chapter{Introdução}

O objetivo deste projeto é construir um sistema de gestão de vendas modular,
de forma a ser capaz de armazenar informações de vendas e relacionar produtos,
clientes e vendas de forma eficiente, aplicando uma arquitetura \textit{Model, View,
Controller} e conhecimentos sobre algoritmos e estruturas de dados. Como objetivo, 
é também necessário garantir o encapsulamento dos dados armazenados de forma
a que não sejam possíveis alterações indevidas por agentes externos.\\
Ao longo deste relatório vamos descrever as nossas abordagens a estes problemas e
apresentar alguns testes de performance do nosso projeto final.

\chapter{Problema}
Três ficheiros são fornecidos contendo informação sobre as transações de uma distribuidora:
\begin{itemize}
    \item O primeiro contém ids de produtos.
    \item Outro, ids de clientes.
    \item O último integra informações sobre cada venda efetuada ao longo de um ano. 
\end{itemize}
Com base nesses ficheiros é preciso responder a 12 queries fornecidas:
\begin{enumerate}
    \item Estatísticas sobre os ficheiros lidos.
    \item Números gerais sobre os dados carregados.
    \item Lista dos códigos de produtos não comprados.
    \item Número de vendas realizadas e clientes distintos, num dado mês e numa dada filial.
    \item Número de produtos distintos e gastos totais mês a mês para um dado cliente.
    \item Determinar mês a mês quantas vezes foi comprado, por quantos clientes e o total faturado de um dado produto.
    \item Produtos mais comprados por um dado cliente e a sua quantidade.
    \item X produtos mais vendidos todo o ano e o número de clientes que o compraram.
    \item Determinar os 3 maiores compradores em cada filial.
    \item X clientes que compraram mais produtos diferentes e quantos.
    \item X clientes que compraram um dado produto e qual o valor gasto.
    \item Calculara a faturação total de um dado produto, mês a mês e filial a filial.
\end{enumerate}

\chapter{Módulos e API}\label{chap:api}

Tendo em conta as características do projeto, concluímos que a arquitetura que melhor
abrangia os critérios pedidos era a Arquitetura do tipo \textit{MVC} (Modelo, 
Apresentação e Controlador).
Um dos critérios que mais fez o grupo gravitar em direção a esta escolha foi 
a modularidade inerente a esta arquitetura.

\section{Model}

É na \textit{Package} \textit{Model} onde está toda a parte de algoritmos e dados, nunca 
esta conhecendo ou dada a conhecer à camada da Apresentação.

\subsection{Cliente}

O módulo de Cliente tem uma API capaz de lidar com a informação de um cliente
individual, e sua respetiva validação. 

\subsection{Clientes}

O módulo de Clientes tem uma API capaz de lidar com a informação de todos os
clientes, respetivo armazenamento e pesquisa. \\ 
Como estrutura principal de armazenamento dos Clientes individuais, após testes 
testar várias estruturas, optamos por utilizar os HashMaps presente na biblioteca
\textit{java.util.Map}.

\subsection{Produto}

À semelhança do módulo de Cliente, o módulo de Produto tem uma API desenhada
para tratar da informação referente a um produto individual, bem como a sua 
validação.

\subsection{Produtos}

No módulo de Produtos, a API está construída de maneira a que seja capaz de
lidar, de forma eficiente com o armazenamento e pesquisa de todos os produtos.\\
Para este módulo, como no módulo de Clientes, decidimos utilizar os mesmos HashMaps
da biblioteca \textit{java.util.Map}.

\subsection{Venda}

No módulo de Venda, temos uma API capaz de dar parse de uma string com o formato 
previamente definido, colocar numa estrutura com os campos necessários de maneira
a preservar a informação, para posteriormente ser tratada pelo módulo de Filiais e
Faturação.

\subsection{Factura}

Neste módulo, a API está definida de forma a conseguir, a partir de uma Venda,
criar/atualizar uma fatura, contendo esta, a faturação relativa a um produto.

\subsection{Faturas}

O módulo de Faturas é capaz de comportar informação sobre toda a faturação,
organizada por produtos e guardada numa Hashtable. Uma estrutura do tipo Faturas
para além de guardar a faturação individual de cada produto, para um mais rápido 
acesso, guarda também valores totais de faturação e número de vendas.

\subsection{Filial}

Para o módulo de filial, está definida uma API capaz de fazer a ligação entre 
clientes e os respetivos produtos por eles comprados, e vice-versa.\\
Uma estrutura do tipo Filial é composta por duas Hashtables, uma que contém
informação relativa a todos os clientes que fizeram compras na dada filial, 
que produtos compraram e respetiva faturação, e uma segunda que contém 
informação relativa a todos os produtos vendidos na dada filial, mais 
concretamente que clientes adquiriram o produto.

\subsection{Constantes}

No módulo constantes, são guardadas todas as informações referentes ao número de
filiais, ficheiros a ler, entre outros, que são lidas de um ficheiro de configs.

\subsection{GestVendasModel}

O módulo GestVendasModel é o módulo que junta todos os acima descritos, 
contendo este uma estrutura
que contém um Catálogo de Produtos, um Catálogo de Clientes, uma estrutura Faturação e 
um array de Filiais. Este módulo faz a ponte entre todos os módulos internos e o 
exterior, sendo este o módulo que ao qual o controlador faz pedidos, e tem 
métodos capazes de responder a todas as queries pedidas.

\section{View}

Esta \textit{package} contém as classes de apresentação dos resultados ao utilizador.

\subsection{GestVendasView}

Esta classe representa os vários Menus e as relações entre eles. Para permitir
conhecer o caminho percorrido até ao menu que se está a observar esta classe
contém uma stack com os menus percorridos.

\subsection{Navigator}

A fim de facilitar a apresentação de determinadas queries foi criada uma classe que 
apresenta uma lista de valores sobre a forma de páginas.\\
De forma a que esta representação se ajuste ao terminal que está a ser utilizado, 
o número de linhas e de colunas que vai ser representado em cada página navegável 
é calculado com base no tamanho do terminal presente. Para tal, a classe Terminal 
é utilizada.

\subsection{Table}

Alguns dos dados obtidos nas queries variam em duas variáveis discretas finitas, como por exemplo
ao mês a mês e filial a filial em simultâneo.\\
Assim, naturalmente, a melhor forma de representar tais dados é através de uma tabela.
Para tal foi desenvolvida uma classe para representar tabelas com um 
\textit{generic type parameter}.\\
Consequentemente, esta classe apenas precisa de duas listas de etiquetas (uma 
para as linhas e outra para as colunas) e uma Lista de Listas com os dados para 
representar uma tabela visualmente apelativa em que cada coluna automaticamente 
adapta o seu tamanho ao seu conteúdo.

\section{Controller}

Cria a ponte entre o View e o Model. Assim, o Controller é o único que conhece a view e o
Model, sendo que tanto a View como o Model apenas conhecem o Controller.

\subsection{GestVendasController}

Esta classe atua como o controlador do nosso projeto. Para além de funcionar como a ponte entre
a lógica e a apresentação também cronometra o tempo demorado pela lógica a responder às queries pedidas
e passa esse valor à \textit{View} para ser apresentado ao utilizador.

\section{Exceptions}

Esta Package contém todas as classes de exceções utilizadas ao longo do projeto.
Estas são usadas para assinalar, por exemplo, uma filial ou um cliente inválido passado
como argumento.

\pagebreak

\section{Utils}

A \textit{package} Utils contém classes que são utilizadas em várias partes do projeto e que não
pertencem a uma parte específica da arquitetura \textit{MVC}.

\subsection{Crono}

A Classe Crono está desenhada para medir o tempo decorrido ao longo da execução do projeto.
Para tal possui dois métodos, o \textit{start} e o \textit{stop} que devem ser chamados no
inicio e no fim da porção de código que se quer cronometrar, respetivamente.
Por fim, para apresentar o resultado ao utilizador, a função \textit{toString} foi implementada
de forma a apresentar os tempos calculados de forma legível.

\subsection{Terminal}

A classe Terminal calcula o tamanho do terminal onde o programa está a correr.
Para tal realiza duas \textit{System Calls} com dois \textit{Fork exec} (uma para obter a altura
e outra para obter a largura do terminal).
Para recalcular o tamanho basta chamar o método \textit{update}.

\subsection{StringBetter}

A Classe StringBetter contém métodos que permitem formatar texto quando este é impresso na
\textit{Shell}. Estes métodos incluem mudar a cor, negrito e itálico e repetir o texto N vezes.\\
Afim de agilizar a utilização da classe, todos os métodos devolvem uma instância da classe
de forma a ser possível o encadeamento de métodos.\\
Assim, se um utilizador quiser escrever \textit{Hello World} a negrito, sublinhado e a
vermelho basta escrever.
\begin{figure}[H]
    \begin{center}
        \begin{verbatim}
        out.println(new StringBetter("Hello World").bold().under().red()
        \end{verbatim}
    \end{center}
\end{figure}

\chapter{Testes e Benchmarks}

Durante a execução do nosso projeto, foram efetuados diversos testes, quer ao nível
de formas de leitura do ficheiro, em particular, \textit{BufferedReader} e \textit{Files},
bem como testes ao nível das diversas implementações das Interfaces \textit{Map},
\textit{Set} e \textit{List}.

\section{Tempos de execução}

Com recurso à classe fornecida \textit{Crono}, efetuamos alguns benchmarks, obtendo
a tabela \ref{tab:benches}, com os tempos médios de execução, com os diversos ficheiros
de vendas fornecidos.\\
Nestes vários benchmarks, testamos diferentes implementações das Interfaces \textit{Map},
nomeadamente, \textit{HashMap} e \textit{TreeMap}, concluindo assim que os tempos de 
que os \textit{HashMap} são muito mais rápidos no carregamento da informação para 
memória, mantendo os tempos das queries praticamente inalterados, como é possível
comprovar na tabela \ref{tab:benchesTH}.\\
Foram também testados tempos de execução das Implementações da Interface \textit{List},
nomeadamente \textit{ArrayList} e \textit{Vector}, visível na tabela \ref{tab:benchesLV}
não havendo diferenças notáveis, acabando assim por optar pela implementação 
mais comum \textit{ArrayList}.

\section{Tempos de Leitura}
Ao nível de leitura, foram efetuados benchmarks para comparar a performance de
\textit{Files} e de \textit{BufferedReader} vistos em no apêndice \ref{chap:brf}. Sendo o
\textit{Files} mais lento e já que o \textit{Files} utiliza internamente um
\textit{BufferedReader}, acabamos por utilizar o \textit{BufferedReader} para a leitura
dos ficheiros com a informação. Testamos também a leitura e validação dos ficheiros
tirando partido do paralelismo intrínseco da JVM8, utilizando streams,
verificando que os tempos de carregamento baixavam para perto de metade, no caso do
ficheiro com 5 milhões de vendas.

\chapter{Conclusão}

Para concluir, conseguimos cumprir todos os requisitos propostos, conseguindo implementar
todos os módulos e estrutura-los como pedido, sendo assim capaz de responder a todas as 
queries da forma que nos pareceu mais eficiente.\\
Como trabalho futuro, gostaríamos de melhorar a maneira da organização da informação 
referente às filiais, de forma a baixar tempo de resposta a queries que utilizam informação
nele presente.

\appendix

\chapter{Diagramas de Classes}
\begin{figure}[H]
    \begin{center}
        \includegraphics[width=1\textwidth]{modelGraph.png}\par
        \caption{Diagrama de Classes do Model}
    \end{center}
\end{figure}

\begin{figure}[H]
    \begin{center}
        \includegraphics[width=1\textwidth]{viewGraph.png}\par
        \caption{Diagrama de Classes do Controller}
    \end{center}
\end{figure}

\begin{figure}[H]
    \begin{center}
        \includegraphics[width=1\textwidth]{controllerGraph.png}\par
        \caption{Diagrama de Classes do View}
    \end{center}
\end{figure}

\chapter{Benchmarks BufferedReader vs Files}\label{chap:brf}

\begin{table}[H]
    \begin{center}
        \begin{tabular}{| c | c | c | c |}
            \hline
            & 1 Milhão & 3 Milhões & 5 Milhões \\
            \hline
            Leitura & 162 & 434 & 750 \\
            \hline
            Parse & 871 & 1884 & 3013 \\
            \hline
            Validação & 1316 & 3120 & 4380 \\
            \hline
            Validação Paralela & 795 & 1535 & 2203 \\
            \hline

        \end{tabular}
        \caption{Tempo (em ms) de tempos de BufferedReader} 
        \label{tab:BF}
    \end{center}
\end{table}

\begin{table}[H]
    \begin{center}
        \begin{tabular}{| c | c | c | c |}
            \hline
            & 1 Milhão & 3 Milhões & 5 Milhões \\
            \hline
            Leitura & 173 & 480 & 773 \\
            \hline
            Parse & 861 & 1948 & 3017 \\
            \hline
            Validação & 1139 & 2755 & 4380 \\
            \hline
            Validação Paralela & 828 & 1572 & 2222 \\
            \hline

        \end{tabular}
        \caption{Tempo (em ms) de tempos Files}
        \label{tab:Files}
    \end{center}
\end{table}


\chapter{Tabela de Tempos de Execução}

\begin{table}[H]
    \begin{center}
        \begin{tabular}{| c | c | c | c |}
            \hline
            & 1 Milhão & 3 Milhões & 5 Milhões \\
            \hline
            Load Time & 2002 & 4522 & 7359 \\
            \hline
            Query 1 & 68 & 32 & 31 \\
            \hline
            Query 2 & 39 & 80 & 119 \\
            \hline
            Query 3 & 55 & 20 & 14 \\
            \hline
            Query 4 & 1 & 1 & 1 \\
            \hline
            Query 5 & 13 & 22 & 17 \\
            \hline
            Query 6 & 950 & 2306 & 3730 \\
            \hline
            Query 7 & 71 & 111 & 72 \\
            \hline
            Query 8 & 240 & 1115 & 1526 \\
            \hline
            Query 9 & 9 & 14 & 10 \\
            \hline
            Query 10 & 45 & 100 & 230 \\
            \hline

        \end{tabular}
        \caption{Tempo (em ms) das queries para um dado número de vendas}
        \label{tab:benches}
    \end{center}
\end{table}

\begin{table}[H]
    \begin{center}
        \begin{tabular}{| c | c | c | c |}
            \hline
            & 1 Milhão & 3 Milhões & 5 Milhões \\
            \hline
            Load Time & 2117 & 4824 & 7270 \\
            \hline
            Query 1 & 21 & 32 & 34 \\
            \hline
            Query 2 & 78 & 86 & 117 \\
            \hline
            Query 3 & 5 & 20 & 18 \\
            \hline
            Query 4 & 1 & 1 & 1 \\
            \hline
            Query 5 & 11 & 26 & 18 \\
            \hline
            Query 6 & 858 & 2405 & 3942 \\
            \hline
            Query 7 & 67 & 121 & 57 \\
            \hline
            Query 8 & 267 & 900 & 1418 \\
            \hline
            Query 9 & 7 & 5 & 10 \\
            \hline
            Query 10 & 33 & 104 & 176 \\
            \hline

        \end{tabular}
        \caption{Tempo (em ms) das queries para um dado número de vendas, utilizando
        \textit{Vector} em vez de \textit{ArrayList}}
        \label{tab:benchesLV}
    \end{center}
\end{table}

\begin{table}[H]
    \begin{center}
        \begin{tabular}{| c | c | c | c |}
            \hline
            & 1 Milhão & 3 Milhões & 5 Milhões \\
            \hline
            Load Time & 5060 & 16100 & 25369 \\
            \hline
            Query 1 & 74 & 32 & 27 \\
            \hline
            Query 2 & 69 & 88 & 94 \\
            \hline
            Query 3 & 45 & 10 & 18 \\
            \hline
            Query 4 & 1 & 2 & 1 \\
            \hline
            Query 5 & 48 & 15 & 14 \\
            \hline
            Query 6 & 929 & 2528 & 3690 \\
            \hline
            Query 7 & 62 & 68 & 80 \\
            \hline
            Query 8 & 269 & 816 & 1430 \\
            \hline
            Query 9 & 7 & 5 & 10 \\
            \hline
            Query 10 & 50 & 111 & 155 \\
            \hline

        \end{tabular}
        \caption{Tempo (em ms) das queries para um dado número de vendas, utilizando
        \textit{TreeMap} em vez de \textit{HashMap}}
        \label{tab:benchesTH}
    \end{center}
\end{table}


\end{document}
